\documentclass[a4paper,8pt]{extarticle}
\usepackage{amssymb,amsmath,amsthm,amsfonts}
\usepackage{multicol,multirow}
\usepackage{calc}
\usepackage{ifthen}
\usepackage{tabularx}
\usepackage[utf8]{inputenc}
\usepackage[landscape]{geometry}
\usepackage[colorlinks=true,citecolor=blue,linkcolor=blue]{hyperref}
\usepackage{accents}
\newcommand{\vect}[1]{\accentset{\rightharpoonup}{#1}}

\ifthenelse{\lengthtest { \paperwidth = 11in}}
    { \geometry{top=.5in,left=.5in,right=.5in,bottom=.5in} }
	{\ifthenelse{ \lengthtest{ \paperwidth = 297mm}}
		{\geometry{top=1cm,left=1cm,right=1cm,bottom=1cm} }
		{\geometry{top=1cm,left=1cm,right=1cm,bottom=1cm} }
	}
\pagestyle{empty}
\makeatletter
\renewcommand{\section}{\@startsection{section}{1}{0mm}%
                                {-1ex plus -.5ex minus -.2ex}%
                                {0.5ex plus .2ex}%x
                                {\normalfont\large\bfseries}}
\renewcommand{\subsection}{\@startsection{subsection}{2}{0mm}%
                                {-1explus -.5ex minus -.2ex}%
                                {0.5ex plus .2ex}%
                                {\normalfont\normalsize\bfseries}}
\renewcommand{\subsubsection}{\@startsection{subsubsection}{3}{0mm}%
                                {-1ex plus -.5ex minus -.2ex}%
                                {1ex plus .2ex}%
                                {\normalfont\small\bfseries}}
\makeatother
\setcounter{secnumdepth}{0}
\setlength{\parindent}{0pt}
\setlength{\parskip}{0pt plus 0.5ex}
% -----------------------------------------------------------------------

\title{Analiza 2}

\begin{document}

\raggedright
\footnotesize

\begin{multicols}{4}
\setlength{\premulticols}{1pt}
\setlength{\postmulticols}{1pt}
\setlength{\multicolsep}{1pt}
\setlength{\columnsep}{2pt}

\section{Odvodi}
\setlength{\tabcolsep}{0.5em}{\renewcommand{\arraystretch}{1.2}
\begin{tabular}{ | r | l | }
    \hline
    \emph{funkcija} & \emph{odvod}\\\hline
    $c$ & $0$ \\ \hline
    $x^n$ & $nx^{n-1}$ \\ \hline
    $a^x$ & $a^x\ln{a}$ \\\hline
    $\frac{a^x}{\ln a}$ & $a^x$\\\hline
    $x^x$ & $x^x(1+\ln{x})$ \\\hline
    $\ln(x)$ & $\frac{1}{x}$ \\\hline
    $\log_{a}(x)$ & $\frac{1}{x\ln(a)}$ \\\hline
    $\sin(x)$ & $cos(x)$ \\\hline
    $\cos(x)$ & $-sin(x)$ \\\hline
    $\tan(x)$ & $\frac{1}{cos^2(x)}$ \\\hline
    $\cot(x)$ & $-\frac{1}{sin^2(x)}$ \\\hline
    $\arcsin(x)$ & $\frac{1}{\sqrt{1-x^2}}$ \\\hline
    $\arccos(x)$ & $-\frac{1}{\sqrt{1-x^2}}$ \\\hline
    $\arctan(x)$ & $\frac{1}{1+x^2}$ \\\hline
    $\textrm{arccot}(x)$ & $-\frac{1}{1+x^2}$ \\\hline
    $\textrm{sh}(x) = \frac{e^x - e^{-x}}{2}$ & $\textrm{ch}(x)$\\\hline
    $\textrm{ch}(x) = \frac{e^x + e^{-x}}{2}$ & $\textrm{sh}(x)$\\\hline
    $\textrm{th}(x) = \frac{\textrm{sh}(x)}{\textrm{ch}(x)}$ & $\frac{1}{\textrm{ch}^2(x)}$\\\hline
    $\textrm{cth}(x) = \frac{1}{\textrm{th}(x)}$ & $-\frac{1}{\textrm{sh}^2(x)}$\\\hline
    $\textrm{arsh}(x) = \ln(x+\sqrt{x^2+1})$ & $\frac{1}{\sqrt{1+x^2}}$\\\hline
    $\textrm{arch}(x) = \ln(x+\sqrt{x^2-1})$ & $\frac{1}{\sqrt{1-x^2}}$\\\hline
    $\textrm{arth}(x) = \frac{1}{2}\ln{\frac{1+x}{1-x}}$ & $\frac{1}{(1+x)(1-x)}$\\\hline
\end{tabular}
}

\subsubsection{Pravila za odvajanje}
\setlength{\tabcolsep}{0.5em}{\renewcommand{\arraystretch}{1.2}
\begin{tabular}{ c  c }
    \emph{funkcija} & \emph{odvod}\\
    $f(x)\pm g(x)$ & $f'(x)\pm g'(x)$ \\
    $f(x)\cdot g(x)$ & $f'(x)\cdot g(x) + f(x)\cdot g'(x)$ \\
    $\frac{f(x)}{g(x)}$ & $\frac{f'(x)\cdot g(x) - f(x) \cdot g'(x)}{g^2(x)}$ \\
    $f(g(x))$ & $f'(g(x)) \cdot g'(x)$ \\
    $f^{-1}(x)$ & $\frac{1}{f'(f^{-1}(x))}$
\end{tabular}
}

\section{Integracijske metode}
\subsubsection{Uvedba nove spremenljivke}
\[ \int f(x)\,dx \underbrace{=}_{x=g(t)} \int f(g(t))g'(t)dt\]

\[u = g(x) \implies du = g'(x) dx \implies dx = \frac{du}{g'(x)}\]

\subsubsection{Perpartes}
\[\int u(x)\,v'(x)\,dx = u(x)v(x)-\int v(x)\,u'(x)\,dx\]

\subsubsection{Integral racionalne funkcije}
Z deljenjem zapišemo racionalno funkcijo $R(x)$ v obliki $p(x) + \frac{r(x)}{q(x)}$, kejr je $r$ nižje stopnje od $q$.

Polinom $q$ rezcepimo na linearne in nerazcepne kvadratne faktorje.

Funkcijo $\frac{p(x)}{q(x)}$ zapišemo kot vsoto parcialnih ulomkov:
\[\frac{1}{(x-a)^k} \leadsto \frac{A_1}{(x-a)}+\frac{A_2}{(x-a)^2} + ... + \frac{A_k}{(x-a)^k} \]
\[\frac{1}{(x^2+bx+c)^l} \leadsto \frac{B_1 + C_1x}{(x^2+bx+c)} + ... + \frac{B_l + C_lx}{(x^2+bx+c)^l} \]
Parcialne ulomke posamično integriremo:
\[ \int \frac{dx}{ax^2+bx+c} = \frac{1}{a \omega} \arctan \left( \frac{2ax + b}{2a \omega}\right);\ \omega = \frac{c}{a} - \left(\frac{b}{2a}\right)^2\]
\[ \int \frac{px+q}{\underbrace{ax^2+bx+c}_{t}}dx = \frac{p}{2a} \ln \left| t \right| + \left( q-\frac{pb}{2a} \right) \int \frac{dx}{t}\]
\[ \int \frac{px+q}{(ax^2+bx+c)^n}dx = \frac{p}{2a}\frac{t^{1-n}}{1-n} + \left( q-\frac{pb}{2a} \right) \int \frac{dx}{t^n}\]
\[ \int \frac{dx}{(ax^2+bx+c)^n} = \frac{1}{a^n \omega^n} I_n\]

\begin{equation*}
    \begin{aligned}
        I_n = \int \frac{dx}{(t^2+1)^n} &&
        I_1 = \arctan t
    \end{aligned}
\end{equation*}

\[ I_n = I_{n-1}\left(1-\frac{1}{2(n-1)}\right) + \frac{t}{2(n-1)(t^2+1)^{n-1}}\]

\subsubsection{Integrali trigonometričnih funkcij}
Integrale z trigonometričnimi funkcijami z univerzalno trigonometrično substitucijo prevedemo na integral racionalne funkcije.
\begin{equation*}
    \begin{aligned}
        \textmd{tan}\frac{x}{2} = t&&
        dx = \frac{2dt}{1+t^2}&&
        \textmd{cos}x = \frac{1-t^2}{1+t^2}&&
        \textmd{sin}x = \frac{2t}{1+t^2}&&
    \end{aligned}
\end{equation*}

\subsubsection{Integral iracionalne funkcije}
Integrale tipa $\int \frac{p(x) dx}{\sqrt{ax^2+bx+c}}$ rešujemo na naslednji način:

\begin{itemize}
    \item Če je $p$ konstanten, integral (s substitucijo) prevedemo na enega izmed:
    \[\int \frac{dx}{\sqrt{a^2-x^2}} = \arcsin\left( \frac{x}{a} \right) + C;\quad a > 0\]
    \[\int \frac{dx}{\sqrt{x^2-a^2}} = \ln \left| x + \sqrt{x^2 - a^2} \right| + C;\quad a > 0\]
    \[\int \frac{dx}{\sqrt{x^2+a^2}} = \ln \left( x + \sqrt{x^2 + a^2} \right) + C;\quad a > 0\]
    \item Če je $p$ poljuben polinom, uporabimo nastavek: \\\ \\
    $\int \frac{p(x)}{\sqrt{ax^2+bx+c}} dx  = \widetilde{p}(x) \sqrt{ax^2+bx+c} + \int \frac{C}{\sqrt{ax^2+bx+c}}dx $\\
    $C$ je konstanta, $\widetilde{p}$ pa polinom, ki ima stopnjo 1 manjšo kot $p$. Koeficiente polinoma $\widetilde{p}$ in konstanto $C$ dobimo z odvajanjem zgornje enačbe.
\end{itemize}

\section{Uporaba integrala}

\textbf{Računanje površine ravninskih likov pod krivulijo}
\[
    p = \int^b_a f(x)\ dx
\]

\textbf{Dolžina ravninske krivulije}
\[ s = \int^b_a \sqrt{1+f'(x)^2}\ dx \]

\textbf{Prostornina in površina vrtenine} (vrtimo okoli $x$ osi)
\[ V = \pi \int^b_a f(x)^2\ dx\]
\[ P = 2\pi \int^b_a f(x)\sqrt{1+(f'(x))^2}\ dx\]

\textbf{Prostornina vrtenine} (vrtimo okoli $y$ osi)
\[ V = 2\pi \int^b_a x f(x)\ dx \]

\textbf{Težišče ravninskih likov}
\[ y_T = \frac{\int^b_a f(x)^2 \ dx}{2p}\]

Doložina poti, ki jo pri vrtenju za $360^\circ$ opiše težišče je $2\pi y_T$.
\[ 2\pi y_T p = \pi \int^b_a f(x)^2 \ dx = V\]


\textbf{Težišče ravninske krivulije}
\[ y_T = \frac{\int^b_a f(x) \sqrt{1+(f'(x))^2} \ dx}{s}\]

\textbf{Dolžina ravninske krivulije}
\[ y_T = \frac{\int^b_a f(x) \sqrt{1+(f'(x))^2}\ dx}{s} \]

\section{Parametrično podane krivulje}
Enačba krivulje v $\mathbb{R}^3$ je oblike
\[ \vect{r} : [a,b] \to \mathbb{R}^3 \]
\[ \vect{r}(t) = (x(t), y(t), z(t)) \in \mathbb{R}^3 \]

\subsubsection{Krivulje v polarnih koordinatah}
\[r = r(\varphi)\]
\[x(\varphi) = r(\varphi) \cos \varphi\]
\[y(\varphi) = r(\varphi) \sin \varphi\]
\[ \vect{r} (\varphi) = ( r(\varphi) \cos \varphi, r(\varphi) \sin \varphi)\]

\subsubsection{Tangenta parametrične krivulije}
Tangenta je kar vektor, ki ga dobimo z odvajanjem parametrične enačbe.
\[ \vect{r}(t) = (x(t), y(t), z(t))\]
\[ \dot{\vect{r}}(t) = (\dot{x}(t), \dot{y}(t), \dot{z}(t))\]

\subsubsection{Doložina parametrično podane krivulije}
Majhen delček krivulije ima dolžino $ds = \| \vect{r}(t+dt) - \vect{r}(t)\| = \|\dot{\vect{r}}(t)\| $. 
Doložina večjega dela krivulje je potem integral teh delčkov:
\[s = \int^b_a \|\dot{\vect{r}}(t)\|\ dt\]

\subsubsection{Doložina krivulije v polarnih koordinatah}
\[s = \int^b_a \sqrt{\dot{r}(\varphi)^2 + r(\varphi)^2}\ d\varphi\]

\subsubsection{Naravna parametrizacija}
Pot $\vect{r} : I \to \mathbb{R}^3$ je naravno parametrizirana, če je $\forall t\in I\ :\ |\dot{\vect{r}}(t)| = 1$.\\
Vsako pot lahko \textbf{naravno reparametriziramo}. Za nek $a\in I$ definiramo funkcijo
\begin{equation*}
    \begin{aligned}
        s:\ & I \to J & s(t) =& \int_a^t |\dot{\vect{r}}(\tau)|d\tau\\
    \end{aligned}
\end{equation*}
z $t : J \to I$ označimo izverz od s $s$. Potem je $\vect{\varphi}(s) : J \to \mathbb{R}$ dana s predpisom $\vect{\varphi}(s) = \vect{r}(t(s))$ \emph{naravno parametrizirana pot}.

\subsubsection{Frenetove formule}
Naj bo $\vect{r}(s)$ naravna parametrizacija.
\begin{equation*}
    \begin{aligned}
        \underbrace{\vect{T} = \vect{r}\,'}_\textmd{tangenta} &&
        \underbrace{\vect{N} = \frac{\vect{T}'}{|\vect{T}'|}}_\textmd{normala} &&
        \underbrace{\vect{B} = \vect{T} \times \vect{N}}_\textmd{binormala}
    \end{aligned}
\end{equation*}
$\kappa = |\vect{T}'| = |\vect{r}\,''|\ \dots $ fleksijska ukrivljenost\\\ \\
$\tau = \frac{(\vect{r}\,' \times \vect{r}\,'')\vect{r}\,'''}{|\vect{r}\,''|^2}\ \dots $ torzijska ukrivljenost

Če je $\forall s \in I\ :\ \kappa(s) \neq 0$ so vektorji $\vect{T},\vect{N},\vect{B}$ dobro definirani in veljajo \textbf{Frenetove formule}:
\begin{equation*}
    \begin{aligned}
        \vect{T}' = \kappa \vect{N}\quad &&
        \vect{N}' = \tau\vect{B}-\kappa \vect{T} && \quad
        \vect{B}' = -\tau \vect{N}
    \end{aligned}
\end{equation*}

\subsubsection{Frenetova baza in ukrivljenost v poljubni parametrizaciji}
Naj bo $\vect{r}(t)$ poljubna regularna pot z neničelno fleksijska ukrivljenostjo.
\begin{equation*}
    \begin{aligned}
        \vect{T} = \frac{\dot{\vect{r}}}{|\dot{\vect{r}}|} \quad &&
        \vect{B} = \frac{\dot{\vect{r}} \times \ddot{\vect{r}}}{|\dot{\vect{r}} \times \ddot{\vect{r}}|} && \quad
        \vect{N} = \vect{B} \times \vect{T}\\
    \end{aligned}
\end{equation*}
\begin{equation*}
    \begin{aligned}
        \kappa = \frac{|\,\dot{\vect{r}} \times \ddot{\vect{r}}\,|}{|\,\dot{\vect{r}}\,|^3} && \quad
        \tau = \frac{\big(\,\dot{\vect{r}} \times \ddot{\vect{r}}\,\big)\dddot{\vect{r}}}{|\,\dot{\vect{r}} \times \ddot{\vect{r}}\,|^2}
    \end{aligned}
\end{equation*}

\section{Metrični prostor}
Preslikava $d:M\to M$ je \textbf{metrika} na množici $M$, če velja:
\begin{itemize}
    \item $d(x,y) \geq 0$ in $d(x,y) = 0 \Leftrightarrow x = y$
    \item $d(x,y) = d(y,x)$
    \item $d(x,z) \leq d(x,y) + d(y,z)$
\end{itemize}
\textbf{Metrični prostor} je par $(M,d)$.\\
\textbf{Normiran prostor} je \emph{vektorski prostor} $V$ opremljen z normo.
\textbf{Norma} je preslikava $\|\,\| : V \to \mathbb{R}^+$, ki ima lastnosti:
\begin{itemize}
    \item $\|v\| = 0 \Rightarrow v = 0$
    \item $\|\alpha v \| = |\alpha| \|v\|$
    \item $ \| w + v \| \leq \| w \| + \| v \| $
\end{itemize}
Potem je $V$ tudi metrični prostor z $d(w,v) = \| w-v \|$.
\subsubsection{Odprte in zaprte množice}
\textbf{Odprta krogla} $B(a,r) = \left\{ x \in M : d(a,x) < r \right\}$\\
\textbf{Zaprta krogla} $\overline{B}(a,r) = \left\{ x \in M : d(a,x) \leq r \right\}$\\
Množiva $U \subseteq M$ je \textbf{odprta}, če 
\[\forall a \in U\ \exists \varepsilon > 0 : B(a,\varepsilon) \subseteq U \]
Množica $U \subset M$ je \textbf{zaprta}, če je $U^C$ odprta.\\
\textbf{Zaprtje} $\bar{A}$ množice $A$ je najmanjša množica v $M$, ki vsebuje $A$. \\
\textbf{Notranjost} $\mathring{A}$ je največja odprta množiva vsebovana v $A$.\\
\subsubsection{Rob množice}
Naj bo $A \subseteq M$.
Točka $a\in A$ je \textbf{robna}, če vsaka odprta krogla $B(a,\varepsilon)$ seka $A$ in $A^C$. \\
\textbf{Rob} $ \partial A $ je množica vseh robnih točk množice $A$.
\subsubsection{Zaporedja}
Zaporedje ($a_n$) z limito v metričenem porstoru ($(M,d)$) je \textbf{konvergentno} z limito $a \in M$, če velja:
\[ \lim_{n \to \infty} d(a_n, a) = 0 \]
\[ \forall \varepsilon > 0\ \exists n_0 \in \mathbb{N} : n \geq n_0 \Rightarrow d(a_n, a) < \varepsilon \]
Zaporedje ($a_n$) je \textbf{Cauchyevo}, če velja:
\[ \forall \varepsilon > 0\ \exists n_0 \in \mathbb{N} : n,m \geq n_0 \Rightarrow d(a_n, a_m) < \varepsilon \]
Vsako konvergentno zaporedje je tudi Cauchyevo. Obratno pa je res samo, če je metrični prostor \textbf{poln}. ($\mathbb{R}^n$ je poln)
\subsubsection{Kompaktnost}
Točka $s \in M$ je \textbf{stekališče} zaporedja $a_n$ v metričnem prostoru $(M,d)$, če $\forall \varepsilon > 0$ v krogli $B(s,\varepsilon)$ neskončno mnogo členov.

Limita je tudi stekališče, obratno pa ni vedno res.

Metrični prostor $M$ je \textbf{kompakten}, če ima vsako zaporedje iz $M$ stekališče v $M$. Podmnožica $U \subseteq M$ je kompaktna, če je kompaktna kot metrični prostor za metriko, ki jo podeduje iz $M$.

Vsak kompakten metrični prostor je \textbf{poln}.

Podmnožica metričenga prostora je \textbf{omejena}, če je če je vsebovana v kaki krogli.

\emph{Heine-Borel} Podmnožica v $\mathbb{R}^n$ je \textbf{kompaktna} $\Leftrightarrow$ je zaprta in omejana. $\langle \textit{vstavi vic o fiziku, ki je čokolado} \rangle$

\subsubsection{Zveznost preslikave}
Preslikava $f:(M_1, d_1) \to (M_2, d_2)$ je zvezna v točki $a$, če
\[\forall \varepsilon > 0\ \exists \delta > 0\ \forall x \in M_1 : d_1(x,a)<\delta \Rightarrow d_2(f(x),f(a))<\varepsilon)\]
\[\forall \varepsilon > 0\ \exists \delta > 0 : f(B(a,\delta)) \subseteq B(f(a), \varepsilon)\]
Preslikava je \textbf{zvezna}, če je zvezna v vsaki točki.

Preslikava $f$ je zvezna v točki $a \in M_1$ $\Leftrightarrow$ za vsako zaporedje $a_n$ v $M_1$, ki konvergira proti $a$, konvergira $f(a_n)$ proti $f(a)$.

\subsubsection{Enakomerna zveznost preslikave}
Preslikava $f:(M_1, d_1) \to (M_2, d_2)$ je enakomerno zvezna, če
\[\forall \varepsilon > 0\ \exists \delta > 0\ \forall x,y \in M_1 : d_1(x,y)<\delta \Rightarrow d_2(f(x),f(y))<\varepsilon\]
Vsaka enakomerno zvezna preslikava je zvezna, obratno pa je res samo če je $(M_1, d_1)$ kompakten.
\subsubsection{Skrčitve}
Preslikava $f:(M, d) \to (M, d)$ je \textbf{skrčitev}, če
\[d(f(x), f(y)) \leq q d(x,y) \quad \textmd{ za nek }q \in [0,1)\]
\subsubsection{Banachovo načelo}
Naj bo $(M,d)$ poln in $f:M\to M$ skrčitev. Potem obstaja natanko en $x_0 \in M$ za katerega je $f(x_0) = x_0$.
\end{multicols}
\end{document}